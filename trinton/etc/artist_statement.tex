\documentclass[11pt]{article}
\usepackage{fontspec}
\usepackage[utf8]{inputenc}
\setmainfont{Bodoni 72 Book}
\usepackage[paperwidth=8.5in,paperheight=11in,margin=1in,headheight=0.0in,footskip=0.5in,includehead,includefoot,portrait]{geometry}
\usepackage[absolute]{textpos}
\TPGrid[0.5in, 0.25in]{23}{24}
\parindent=0pt
\parskip=12pt
\usepackage{nopageno}
\usepackage{graphicx}
\graphicspath{ {./images/} }
\usepackage{amsmath}
\usepackage{tikz}
\newcommand*\circled[1]{\tikz[baseline=(char.base)]{
            \node[shape=circle,draw,inner sep=1pt] (char) {#1};}}

\begin{document}

\vspace*{1\baselineskip}

\begingroup
\begin{center}
\huge Artist Statement
\end{center}
\endgroup

\vspace*{2\baselineskip}

\begingroup
\begin{center}
\leftskip0.5in
To give meaning to experience is to create a symbol. Symbols can take an infinite variety of forms, such as concepts, objects, entities, or rituals (which are symbol-creating symbols), to name only a handful. This creation of symbols, and the accumulation of symbols within a larger system of meaning, is the essence of spirituality. It is the essence of the occult, as far as I am concerned. It seems deeply human to, burdened with a recursive consciousness, vitalize the struggle by simultaneously practicing and interrogating one's agency in themself, and their surroundings. This process need not be supernatural, nor of spirits, nor metaphysical. In fact, I believe that the transcendent capabilities of natural and physical phenomena, symbols, and rituals, are of a dire and consequential nature. \\ I am overwhelmed by the infinity of knowledge that exists in the universe, and deeply impressed by the capabilities of language to describe and create this knowledge. I am also deeply aware of the limitations of language. There seems to exist non-semantic knowledge, which is only attainable via its own experience. This is the shamanic dance in the waterfall, these are the objects on the pagan alter and the occurrences within and around once the circle is cast, this is Gautama Buddha sitting beneath the Bodhi Tree. This is the death of the ego, the birth of the unlimited self, experiences wherein those who participate return to the plane of language and express its ineptitude at communicating what they now know. These are not knowledges of language, they are knowledges of experience. \\ My work is a ritual, a construction of and for the creation of symbols, laden with the pursuit of non-semantic knowledge. It seeks not only to discover, but to create, to discourse on this plane. It is for all participants to take a piece of, to bring their own pieces, and to build knowledges of experience from whatever they have, and whatever they find on the plane. \\ I use symbols- marks on a page, lines of code or graphic objects on a screen, the routing of signals through circuitry- to create objects to be imbued with meaning by the ritual of performing, listening, perceiving and understanding. These objects have ever-transforming identities, behaviors, and relationships to one another. Their meaning is transformed by time and behavior, by juxtaposition, by melting in, out, and with one another. It's very important to me that they have no ethic. If they contest one another, neither good nor evil prevail. Rather, knowledges are accumulated by the witness of, and participation in, both the heated and the gentle interactions between these objects whose meanings are always metamorphosing, never arriving. \\ I make scripts of runes and text. I make symbols of sound and meaning. I make performances, I make rituals, I make poetry, and I make music.
\phantom{text} \hfill - Trinton
\end{center}
\endgroup

\end{document}